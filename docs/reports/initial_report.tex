\documentclass{article}
\usepackage[english]{babel}

\begin{document}
\title{\Huge Initial Report \normalsize}
\date{\today}
\author{\LARGE 16 Onions \normalsize \\[5pt] Josef Stark \hspace{20pt} Charlie Groh}
\maketitle
\let\thefootnote\relax\footnotetext{This work is licensed under the Creative
Commons Attribution-ShareAlike 4.0 International License. To view a copy of
this license, visit http://creativecommons.org/licenses/by-sa/4.0/ or send a
letter to Creative Commons, PO Box 1866, Mountain View, CA 94042, USA.}

\section{General}
We are the team ``16 Onions'' consisting of Josef Stark and Charlie Groh
and our goal is to develop a prototype implementation of the Onion module.

\section{Programming Language and Operating System}
Since the anonymity of a person using the VoIP application depends on the
number of other users, it is important that it is as simple as possible to port
the software to nearly every environment. Therefore we decided to use Java 7
as programming language and Linux as operating system. But we think that the
module should be able to run on every operating system, though we will not test
it on other systems than Linux. Another advantage of Java are the built in
safety mechanisms preventing common memory corruption vulnerabilities.

\section{Build System}
As development environment we will use Eclipse on a Debian GNU/Linux system. For
writing the reports LaTeX will be utilized. The exact versions will vary among the
different workstations of the team members.

We think that the Unit-Test-Framework of Java will do a good job in testing
single modules for API- and protocol-conformance. For testing the behaviour of
many instances of the module !!!!!!MISSING!!!!!! will be used.

\section{Libraries}

\section{License}
Our programming work will be released under the GNU Public License, because we
believe it is necessary to release privacy/anonymity related applications as
open source software. Firstly it should be simple for external researchers to
audit the software to find possible security holes. Secondly users should be
able to verify that there are no backdoors or information leakages. Thirdly we
hope that our developement can help others with their own projects and
therefore we decided to use the widespread GPL license.

Due to the same reasons we will place every documentation and report under the
GNU Free Documentation License Version 1 or Creative Commons
Attribution-ShareAlike 4.0 International License depending on which one fits
better.

\section{Previous Programming Experience} 
Both team member learned Java as their first programming language at school and
used it since then in several bigger projects. Therefore we are familiar in
using internal and external libraries, sending and receiving data in various
protocols by using sockets and testing using unit tests.

\section{Work Sharing Strategies}
We will share the programming work by dividing it in smaller pieces and
weighting it with the expected workload. Then the pieces will be assigned to
the team members ensuring that every member gets a similiar workload.
Additionally we will meet at least two times per week to discuss implementation
and integration problems and wrongly weighted pieces.

The reports will be written in the same manner as the programming is done. We
will try to assign every report piece to that team member who programmed the
respective software piece.

\section{Issues and Complains}
Until now we did not discover any issues or complains concerning the project
and hope this will last.
\end{document}
