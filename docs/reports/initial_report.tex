\documentclass{article}
\usepackage[english]{babel}
\usepackage{url}
\begin{document}
\title{\Huge Initial Report \normalsize}
\date{\today}
\author{\LARGE 16 Onions \normalsize \\[5pt] Josef Stark \hspace{20pt} Charlie Groh}
\maketitle
\let\thefootnote\relax\footnotetext{This work is licensed under the Creative
Commons Attribution-ShareAlike 4.0 International License. To view a copy of
this license, visit http://creativecommons.org/licenses/by-sa/4.0/ or send a
letter to Creative Commons, PO Box 1866, Mountain View, CA 94042, USA.}

\section{General}
We are the team ``16 Onions'' consisting of Josef Stark and Charlie Groh,
and our goal is to develop a prototype implementation of the Onion module.

\section{Programming Language and Operating System}
Since the anonymity of a person using the VoIP application depends on the
number of other users, it is important that it is as simple as possible to port
the software to nearly every environment. Therefore we decided to use Java 7
as programming language as there is no need to change or even only recompile Java source code to run it on different platforms. As operating system we use Linux, since both team members are fairly familiar with it. We will not
test it the module on other systems than Linux, since this is the most important one for us and the program behavior shouldn't change on other platforms running the JVM. Another advantage of Java are the built in safety mechanisms preventing common memory corruption vulnerabilities. 
Java is often quoted as being slower than e.g. C or C++, but we are convinced that for our project and with todays hardware, this will not be a concern; far more important would be the efficiency of our implementation in terms of e.g. network usage.

\section{Build System}
As development environment we will use Eclipse on a Debian GNU/Linux system. For
writing the reports, LaTeX will be utilized. The exact versions will vary among
the different workstations of the team members.
Eclipse does a good job at handling dependencies and incremental builds for reduced build times, so most of the time we will just rely on the IDE to do the compiling. But in order to also be able to build the project from the command line (e.g. building on a server with no graphical interface available), we will regularly generate an ant build file out of the Eclipse project. This file, which will reside in the project root folder, can then be interpreted by the command line tool ant independently of Eclipse.

We think that the Unit-Test-Framework of Java will do a good job in testing
single modules for protocol-conformance. For testing API
conformance we will use \url{https://gitlab.lrz.de/voidphone/testing}, which is
provided by the course instructors. For testing the behaviour of many instances
of the module the network virtualizer Mininet will be used.
In comparison to more low-level languages like C, Java has improved safety quite a bit by e.g. removing the possibility of direct memory access and introducing automatic bounds checking and memory management with garbage collection. This eliminates the need to use memory checkers. \\
To avoid other types of bugs and ensure good quality code, we might use eCobertura\footnote{\url{http://ecobertura.johoop.de}}, JLint\footnote{\url{https://sourceforge.net/projects/jlinteclipse/}} and/or FindBugs\footnote{\url{http://findbugs.sourceforge.net/}} for code coverage, data flow and statical code analysis.

\section{Libraries}
Java 7 already ships a pretty big standard library, which will probably provide most of the functionality required for implementing the Onion module. For advanced functionality we might have to use 3rd party libraries, but whenever possible it would be preferable to avoid this as it reduces portability and ease of use. 
The following functional requirements have been identified so far, along with libraries providing it:
\begin{itemize}
\item TCP communication: java.net.Socket (integrated)
\item UDP communication: java.net.DatagramSocket (integrated)
\item INI config file handling: May be possible with java.util.Properties (integrated), if not, we could use ini4j\footnote{\url{http://ini4j.sourceforge.net}} (3rd party) or implement it ourselves.
\item Transferring data/objects: java.io.Serializable (integrated), \\ X-Stream\footnote{\url{https://x-stream.github.io}} (3rd party but human-readable and more compact), Gson\footnote{\url{https://github.com/google/gson}} (same with JSON) or implementing our own protocol (more effort but probably more efficient)
\item Cryptography: java.security, javax.crypto (integrated) or \\ Bouncy Castle Crypto APIs\footnote{\url{https://www.bouncycastle.org}} (3rd party, more functionality)
\end{itemize}



\section{License}
Our programming work will be released under the GNU Public License, because we
believe it is necessary to release privacy/anonymity related applications as
open source software. Firstly it should be simple for external researchers to
audit the software in order to find possible security holes. Secondly users
should be able to verify that there are no backdoors or information leakages.
Thirdly we hope that our development can help others with their own projects,
and therefore we decided to use the widespread GPL license.

Due to the same reasons we will place every documentation and report under the
GNU Free Documentation License Version 1 or Creative Commons
Attribution-ShareAlike 4.0 International License depending on the use case of
the document.

\section{Previous Programming Experience} 
Both team members learned Java as their first programming language at school and
used it in several bigger projects. Therefore we are familiar in using internal
and external libraries, sending and receiving data in various protocols by
using sockets and testing using unit tests.

\section{Work Sharing Strategies}
We will share the programming work by dividing it in smaller pieces and
weighting it with the expected workload. Then the pieces will be assigned to
the team members ensuring that every member gets a similar workload.
Additionally we will meet at least two times per week to discuss implementation
and integration problems and wrongly weighted pieces.

The reports will be written in the same manner as the programming is done. We
will try to assign every report piece to that team member who programmed the
respective software piece.

\section{Issues and Complains}
Until now we did not discover any issues or complaints concerning the project
and hope this will last.
\end{document}
\grid
